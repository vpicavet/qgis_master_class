%  !TeX  root  =  sigll.tex
\mainmatter
\pagestyle{scrheadings}

\chapter{Présentation}\label{sec:qgis_presentation}

\section{Intervenants}

\begin{description}
\item[Vincent Picavet], OSGEO-fr - Oslandia
\item[Jean-Roc Morreale], OSGEO-fr - CG62
\end{description}

\section{QGIS et historique}

Quantum GIS est un SIG libre débuté en 2002. \qg est utilisable sur la majorité des Unix, Mac OS X et Windows. \qg utilise la bibliothèque logicielle Qt et le langage C++, ce qui ce traduit par une interface graphique simple et réactive.

\qg se veut simple à utiliser, fournissant des fonctionnalités courantes. Le but initial était de fournir un visionneur de données SIG. \qg a, depuis, atteint un stade dans son évolution où beaucoup y recourent pour leurs besoins quotidiens. \qg supporte un grand nombre de formats raster et vecteur, avec le support de nouveaux formats facilité par l'architecture des modules d'extension.

\qg est distribué sous la licence GNU GPL (General Public License). Ceci signifie que vous pouvez étudier et modifier le code source, tout en ayant la garantie d'avoir accès à un programme SIG non onéreux et librement modifiable.
