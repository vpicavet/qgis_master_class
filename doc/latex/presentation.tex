%  !TeX  root  =  sigll.tex
\mainmatter
\pagestyle{scrheadings}

\chapter{Présentation}\label{sec:qgis_presentation}

\section{Intervenants}

\begin{description}
\item[Vincent Picavet], OSGEO-fr - Oslandia
\item[Jean-Roc Morreale], OSGEO-fr - CG62
\end{description}

\section{Présentation du Master Class}

\textit{Sergent Hartman :} Si vous ressortez de chez moi, les louloutes, si vous survivez à mon instruction, vous deviendrez une arme, vous deviendrez un prêtre de la mort implorant la guerre ! Mais en attendant ce moment-là, vous êtes du vomi, vous êtes le niveau 0 de la vie sur terre, vous n'êtes même pas humain, bande d'enfoirés ! Vous n'êtes que du branlomane végétatif, des paquets de merde d'amphibiens, de la chiasse ! Parce que je suis une peau de vache, vous me haïrez; mais plus vous me haïrez et mieux vous apprendrez ! Je suis vache mais je suis réglo ! Aucun sectarisme racial ici : je n'ai rien contre les négros, ritals, youpins ou métèques. Ici, vous n'êtes que des vrais connards et j'ai pour consigne de balancer toutes les couilles de loup qui n'ont pas la pointure pour servir ma chère unité ! Tas de punaises, est-ce que c'est clair ?!

\textit{Tous les soldats :} Chef, oui, Chef !

\section{QGIS et historique}

Quantum GIS est un SIG libre débuté en 2002. \qg est utilisable sur la majorité des Unix, Mac OS X et Windows. \qg utilise la bibliothèque logicielle Qt et le langage C++, ce qui ce traduit par une interface graphique simple et réactive.

\qg se veut simple à utiliser, fournissant des fonctionnalités courantes. Le but initial était de fournir un visionneur de données SIG. \qg a, depuis, atteint un stade dans son évolution où beaucoup y recourent pour leurs besoins quotidiens. \qg supporte un grand nombre de formats raster et vecteur, avec le support de nouveaux formats facilité par l'architecture des modules d'extension.

\qg est distribué sous la licence GNU GPL (General Public License). Ceci signifie que vous pouvez étudier et modifier le code source, tout en ayant la garantie d'avoir accès à un programme SIG non onéreux et librement modifiable.