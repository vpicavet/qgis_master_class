%  !TeX  root  =  sigll.tex
\chapter{Exercices}

\section{Importer des données}

\subsection{Importer des vecteurs}

Ouvrez le menu \mainmenuopt{Couches} \arrow \dropmenuopttwo{mActionAddOgrLayer.png}{Ajouter une couche vecteur}

Cliquez sur le bouton \button{Parcourir} et sélectionnez dans la liste déroulante le format \textbf{ESRI Shapefile}

Sélectionnez dans le dossier ressources/vecteurs les fichiers \filename{eau.shp} et \filename{jardin.shp}

Cliquez sur \button{Ouvrir} pour finaliser l'opération.

Répétez la manipulation en sélectionnant cette fois \textbf{Mapinfo} comme format et le fichier \filename{bati mapinfo.mif}

\subsection{Importer des rasters}

Ouvrez le menu \mainmenuopt{Couches} \arrow \dropmenuopttwo{mActionAddRasterLayer.png}{Ajouter une couche raster}, 

Sélectionnez le format \textbf{GeoTIFF} puis les fichiers \filename{srtm bassin parisien.tif} et \filename{srtm ombrage.tif} 

Cliquez sur \button{Ouvrir}.

\section{Bases de l'interface}

\begin{figure}[ht]
   \centering
   \includegraphics[clip=true, width=10cm]{interface}
\end{figure}

\begin{itemize}
\item 1 - légende cartographie : liste les couches chargées dans le projet
\item 2 - canevas : affiche les couches actives
\item 3 - menus : permet l'accès aux fonctions
\end{itemize}

\subsection{Agencer les couches}

Faites un clic droit sur la zone de légende puis choisissez \dropmenuopt{Ajouter un groupe}. Un nouveau dossier apparaît où vous pouvez maintenant glisser et déposer les couches sur l'icône de ce dossier. 

Créez un groupe Vecteurs et un groupe Rasters. Pour changer le nom du groupe, sélectionnez \dropmenuopt{Renommer} dans le menu contextuel du groupe.

\subsection{Centrer l'affichage}

Décochez la case du groupe Rasters afin de ne plus afficher les couches qu'il contient.

Sélectionnez le groupe Vecteurs, faites un clic-droit et cliquez sur \dropmenuopt{Zoomer sur le groupe}.

\section{Utiliser l'interface}\label{sec:ui_use} 

\subsection{Sélectionner des entités}

Sélectionnez la couche \filename{jardin}.

Dans la barre d'outils \textit{Attributs}, cliquez sur l'outil \dropmenuopttwo{mActionSelect}{Sélection d'entités}.

Cliquez sur un objet du canevas, utilisez la touche \keystroke{Ctrl} pour faire une sélection multiple.

Cliquez sur \dropmenuopttwo{mActionDeselectAll}{Désélectionner toutes les entités}.

\subsection{Identification}

Cliquez sur le bouton \dropmenuopttwo{mActionIdentify}{Identifier les entités} puis sur une entité.

Obtenez la surface en dépliant la ligne \textit{(Dérivé)}.

\subsection{Mesurer une longueur, une aire et un angle}

Pour sélectionner un outil de mesure, cliquez sur \includegraphics[width=0.7cm]{mActionMeasure} puis sur l'outil voulu.

\includegraphics[width=0.7cm]{mActionMeasure} 
\qg peut mesurer des distances réelles entre plusieurs points selon un ellipsoïde défini. \par
\includegraphics[width=0.7cm]{mActionMeasureArea} Les aires peuvent aussi être mesurées.
Dans la fenêtre de mesure apparaît la surface totale mesurée. \par
\includegraphics[width=0.7cm]{mActionMeasureAngle}
Le curseur adopte une forme en croix. Cliquez pour dessiner le premier côté de l'angle à mesurer puis bouger le curseur pour dessiner l'angle désiré.

\subsection{Signets spatiaux} \label{sec:bookmarks}

Les signets spatiaux vous permettent de marquer une zone de la carte pour y retourner plus tard.

\minisec{Créer un signet}
Pour créer un signet :
\begin{enumerate}
\item Déplacez-vous sur une zone précise
\item Sélectionnez le menu \mainmenuopt{Vue} > \dropmenuopt{Nouveau signet} ou appuyez sur le bouton \dropmenuopttwo{mActionNewBookmark}{Nouveau signet...}
\item Entrez un nom pour décrire le signet (jusqu'à 255 caractères)
\item Cliquez sur \button{OK} pour ajouter le signet ou sur \button{Annuler} pour sortir de la fenêtre sans l'enregistrer
\end{enumerate}

\minisec{Zoomer sur un signet}
Cliquez sur le bouton \dropmenuopttwo{mActionShowBookmarks}{Montrer les signets}, sélectionnez le signet voulu en cliquant dessus puis sur le bouton \button{Zoomer sur}. Vous pouvez aussi zoomer en opérant un double-clic.

\subsection{Outils d'annotation}

Cliquez sur \includegraphics[width=0.7cm, clip=true]{mActionTextAnnotation} dans la barre d'outils d'attribut puis cliquez sur le canevas.

Faites un double-clic sur l'annotation pour éditer le texte.

\subsection{La table attributaire}

Cliquez sur le bouton \includegraphics[width=0.7cm]{mActionOpenTable} qui permet d'ouvrir la table attributaire ou par un clic droit sur la couche \filename{jardin}. 

\minisec{Sélectionner une entité depuis la table}
Pour une simple recherche par attribut sur une seule colonne, le champ \button{Chercher pour} peut être utilisé. Sélectionnez la colonne \textit{NOM} sur laquelle doit être opérée la recherche depuis la liste déroulante, tapez \textit{Binet} et appuyez sur le bouton \button{Chercher}. 

Cliquez sur le bouton \toolbtntwo{mActionZoomToSelected}{Zoomer la carte sur les lignes sélectionnées}