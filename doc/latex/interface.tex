%  !TeX  root  =  sigll.tex

\chapter{Découverte de QGIS}\label{sec:qgis_start}

\section{Importer des données}\label{sec:import}

Nous allons charger les données de travail en commençant par les données vectorielles. Ouvrez le menu \mainmenuopt{Couches} \arrow \dropmenuopttwo{mActionAddOgrLayer.png}{Ajouter une couche vecteur}, une nouvelle s'ouvre qui permet de sélectionner les sources de données. 

Le type de source \radiobuttonon{Fichier} est sélectionné par défaut, cliquez sur le bouton \button{Parcourir} et déplacez-vous dans le répertoire ressources. Sélectionnez dans la liste déroulante le format \textbf{ESRI Shapefile} pour limiter l'affichage des fichiers présents dans le répertoire à ce type de format. Sélectionnez les fichiers \filename{eau.shp} et \filename{jardin.shp} \footnote{Pour sélectionner plusieurs fichiers d'un coup, il faut maintenir appuyé le bouton Ctrl du clavier} et cliquez sur le bouton \button{Ouvrir} pour revenir à la fenêtre précédente où les fichiers retenus remplissent la case Jeu de données. Cliquez sur \button{Ouvrir} pour finaliser l'opération.

Répétez la manipulation en sélectionnant cette fois \textbf{Mapinfo} comme format et le fichier \filename{bati mapinfo.mif}.

Vouz allez ajouter des fichiers raster via le menu \mainmenuopt{Couches} \arrow \dropmenuopttwo{mActionAddRasterLayer.png}{Ajouter une couche raster}, sélectionnez le format \textbf{GeoTIFF} puis les fichiers \filename{srtm bassin parisien.tif} et \filename{srtm ombrage} puis cliquez sur \button{Ouvrir}.
        
\section{Bases de l'interface}\label{sec:ui_basic} 

\subsection{Légende cartographique}\label{ui_legend}

La zone de légende cartographique est utilisée pour définir la visibilité et l'ordre d'affichage des couches. Une couche se situant au sommet de la liste de cette légende sera affichée au-dessus de celles qui se situent plus bas dans la liste. La boîte à cocher présente à côté de chacune des couches permet de les afficher ou de les cacher.

La légende affiche 5 lignes correspondantes aux 5 sources de données que nous avons ajoutées. Nous allons rassembler les couches vectorielles en créant un groupe et en les y glissant. Pour ce faire, déplacez votre curseur sur la légende, faites un clic droit puis choisissez \dropmenuopt{Ajouter un groupe}. Un nouveau dossier apparaît et vous pouvez maintenant glisser et déposer les couches sur l'icône de ce dossier. 

Il est possible de basculer le mode d'affichage de toutes les couches d'un groupe en décochant seulement le groupe. Pour retirer une couche d'un groupe, il suffit de pointer votre curseur sur elle, de faire un clic droit et de choisir \dropmenuopt{Mettre l'objet au-dessus}. Pour changer le nom du groupe, sélectionnez \dropmenuopt{Renommer} dans le menu contextuel du groupe.

Le contenu du menu contextuel affiché par un clic droit varie si la couche sélectionnée est de type raster ou vecteur.

\subsection{Vue de la carte}\label{ui_vue}

C'est la partie centrale de \qg puisque les cartes y sont affichées ! le contenu qui s'affiche dépend des couches de types raster et vecteur que vous avez choisies de charger (lire les sections suivantes pour savoir comment charger une couche). La vue de la carte peut être modifiée en portant le focus sur une autre région, ou en zoomant en avant ou en arrière. Plusieurs opérations peuvent être effectuées sur la carte comme il est expliqué dans les descriptions des barres d'outils. La vue de la carte et la légende sont étroitement liées — la carte reflète les changements que vous opérez dans la légende.

Vous pouvez utiliser la molette de la souris pour changer le niveau de zoom de la carte. Placez votre curseur dans la zone d'affichage de la carte et faites rouler la molette vers l'avant pour augmenter l'échelle, vers vous pour la réduire. La position du curseur permet de recentrer la vue lors du changement d'échelle. Vous pouvez modifier le comportement de la molette de la souris en utilisant l'onglet \tab{Outils cartographiques} dans le menu \mainmenuopt{Préférences} >\dropmenuopt{Options}.

Vous pouvez utiliser les flèches du clavier pour vous déplacer sur la carte. Placez le curseur sur la carte et appuyez sur la flèche droite pour décaler la vue vers l'Est, la flèche gauche pour la décaler vers l'Ouest, la flèche supérieure vers le Nord et la flèche inférieure vers le Sud. Vous pouvez aussi déplacer la carte en gardant la touche espace appuyée et en bougeant la souris.

\subsection{La barre de menu}
La barre de menu fournit un accès aux différentes fonctionnalités de \qg par le biais de menus hiérarchiques. Les entrées du menu de niveau supérieur et un résumé de certaines options sont listés ci-dessous, avec les icônes des outils correspondants dans la barre d'outils et leurs raccourcis clavier. L'emplacement de ces entrées varie sensiblement suivant le gestionnaire de fenêtre, donc suivant le système d'exploitation. 

\begin{itemize}
\item Fichier : ouverture et sauvegarde de projets et de compositions
\item Éditer : modification des objets
\item Vue : déplacement sur la carte
\item Couche : ajout et modification des couches de données
\item Préférences : modification des propriétés du projet et des préférences générales
\item Extensions : ajout et gestion des extensions
\item Raster : outils raster
\item Vecteur : outils vecteurs
\item Aide : documentation, etc.
\end{itemize}
       
\section{Utiliser l'interface}\label{sec:ui_use} 

\subsection{Sélectionner des entités}\label{sec:selection}

La barre d'outils fournit plusieurs outils de sélection d'entités à partir du canevas de la carte. pour sélectionner une ou plusieurs entités, cliquez  sur \includegraphics[width=0.7cm]{mActionSelect} \footnote{Un clic plus long ou un clic sur la flèche noir pointée vers le bas suffiront.} et choisissez l'outil :

\begin{itemize}
\item Sélection d'entités
\item Sélection d'entités avec un rectangle
\item Sélection d'entités avec un polygone
\item Sélection d'entités à main levée
\item Sélection d'entités selon un rayon
\end{itemize}
               
\subsection{Identification}

Pour obtenir des informations relatives à une entité, on peut procéder directement par la carte en utilisant le bouton \dropmenuopttwo{mActionIdentify}{Identifier les entités}. Un clic gauche sur une entité fait apparaitre une fenêtre listant ses attributs, les actions disponibles \footnote{Une action est un comportement pré-paramétré sur un champ tel que l'ouverture d'un navigateur internet pour champ contenant un lien http.} et des informations dérivées telle que la surface d'un polygone.


\subsection{La table attributaire}

Le bouton \includegraphics[width=0.7cm]{mActionOpenTable} permet d'ouvrir la table attributaire qui affiche les entités de la couche sélectionnée. Chaque ligne représente une entité avec ses attributs répartis dans  plusieurs colonnes. Chaque entité de la table peut être recherchée, sélectionnée, déplacée et éditée.

Vous pouvez aussi y accéder avec un clic droit sur la couche \filename{jardin}. Cela ouvrira une nouvelle fenêtre qui comportera les attributs de toutes les entités de la couche. Le nombre des entités est affiché dans la barre de titre de la table attributaire.

\minisec{Sélectionner une entité depuis la table}
Pour une simple recherche par attribut sur une seule colonne, le champ \button{Chercher pour} peut être utilisé. Sélectionnez la colonne \textit{NOM} sur laquelle doit être opérée la recherche depuis la liste déroulante, tapez \textit{Binet} et appuyez sur le bouton \button{Chercher}. Pour des recherches plus complexes, passez par le bouton \button{Recherche avancée} qui lancera la fenêtre de construction de requêtes SQL.

Les lignes peuvent être sélectionnées en cliquant sur le numéro de ligne placé tout à gauche. La sélection d'une ligne ne cause pas de changement de position du curseur. \textbf{Plusieurs lignes} peuvent être retenues en maintenant la touche \textbf{Ctrl}. \textbf{Une sélection continue} s'effectue en gardant appuyée la touche \textbf{Shift} et en cliquant sur une nouvelle ligne, toutes les lignes entre la première sélection et la dernière seront prises.

\textbf{Une ligne sélectionnée} représente tout les attributs d'une entité, la table attributaire reflète tous les changements qui seront faits sur la carte et vice versa. Un changement de sélection depuis la table d'attributs provoque également un changement de sélection sur la carte et la sélection d'entités d'une couche différente signifie que d'autres lignes ont été sélectionnées.

Vous pouvez faire un tri sur les colonnes en cliquant sur l'en-tête. Une petite flèche indique l'ordre de tri (une flèche pointant vers le bas indiquera un tri descendant).

\begin{itemize}[label=--]
\item \toolbtntwo{mActionOpenTable}{Desélectionner tout}
\item \toolbtntwo{mActionSelectedToTop}{Déplacer la sélection au sommet}
\item \toolbtntwo{mActionInvertSelection}{Inverser la sélection}
\item \toolbtntwo{mActionCopySelected}{Copier les lignes sélectionnées dans le presse-papier} ou \keystroke{Ctrl-C}
\item \toolbtntwo{mActionZoomToSelected}{Zoomer la carte sur les lignes sélectionnées} ou \keystroke{Ctrl-J}
\item \toolbtntwo{mActionToggleEditing}{Activer le mode d'édition} pour modifier les valeurs des attributs
\item \toolbtntwo{mActionDeleteSelected}{Effacer les entités sélectionnées}
\item \toolbtntwo{mActionNewAttribute}{Nouvelle colonne} pour les couches OGR (>=1.6) et PostGIS
\item \toolbtntwo{mActionDeleteAttribute}{Effacer une colonne}, uniquement pour les couches PostGIS
\item \toolbtntwo{mActionCalculateField}{Ouvrir la calculatrice de champ}
\end{itemize}
               
\subsection{Mesurer une longueur, une aire et un angle}

Pour sélectionner un outil de mesure, cliquez sur \includegraphics[width=0.7cm]{mActionMeasure} puis sur l'outil voulu.

\includegraphics[width=0.7cm]{mActionMeasure} 
\qg peut mesurer des distances réelles entre plusieurs points selon un ellipsoïde défini. Pour le configurer, allez dans le menu \mainmenuopt{Préférences} > \dropmenuopt{Options} puis dans l'onglet \tab{Outils cartographiques} et choisissez l'ellipsoïde approprié. Vous pouvez également modifier ici la couleur du trait et l'unité de mesure (mètre ou pied). Cet outil permet de placer des points sur la carte. La longueur de chaque segment s'affiche dans la fenêtre de mesure ainsi que la longueur cumulée totale. Pour stopper les mesures, faites un clic droit. \par
\includegraphics[width=0.7cm]{mActionMeasureArea} Les aires peuvent aussi être mesurées.
Dans la fenêtre de mesure apparaît la surface totale mesurée. \par
En complément, l'outil de mesure s'accrochera à la couche sélectionnée à partir du moment où celle-ci à un seuil d'accrochage défini. Donc si vous voulez mesurer avec exactitude une ligne ou le contour d'un polygone, spécifiez d'abord un seuil d'accrochage puis sélectionnez la couche. Avec l'outil de mesure, chaque clic de souris se situant dans ce seuil s'accrochera aux entités de cette couche. \par
\includegraphics[width=0.7cm]{mActionMeasureAngle}
Vous pouvez aussi mesurer des angles en sélectionnant l'outil de mesure d'angles. Le curseur adopte une forme en croix. Cliquez pour dessiner le premier côté de l'angle à mesurer puis bouger le curseur pour dessiner l'angle désiré. La mesure est affichée dans une fenêtre de dialogue.

\subsection{Signets spatiaux} \label{sec:bookmarks}
\index{signets}
\index{signets spatiaux|\voir{signets}}

Les signets spatiaux vous permettent de marquer une zone de la carte pour y retourner plus tard.

\minisec{Créer un signet}
Pour créer un signet :
\begin{enumerate}
\item Déplacez-vous sur la zone concernée
\item Sélectionnez le menu \mainmenuopt{Vue} > \dropmenuopt{Nouveau signet} ou appuyez sur la touche \keystroke{Ctrl-B}
\item Entrez un nom pour décrire le signet (jusqu'à 255 caractères)
\item Cliquez sur \button{OK} pour ajouter le signet ou sur \button{Annuler} pour sortir de la fenêtre sans l'enregistrer
\end{enumerate}

Vous pouvez avoir plusieurs signets portant le même nom.

\minisec{Travailler avec les signets}
Pour utiliser ou gérer les signets allez dans le menu \mainmenuopt{Vue} > \dropmenuopt{Montrer les signets}.
Le dialogue \dialog{Signets géospatiaux} vous permet de rappeler ou d'effacer un signet.
Vous ne pouvez pas modifier le nom d'un signet ou ses coordonnées.

\minisec{Zoomer sur un signet}
Depuis la fenêtre \dialog{Signets géospatiaux}, sélectionnez le signet voulu en cliquant dessus puis sur le bouton \button{Zoomer sur}. Vous pouvez aussi zoomer en opérant un double-clic.

\minisec{Effacer un signet}
Pour effacer un signet depuis la fenêtre \dialog{Signets géospatiaux}, cliquez dessus puis sur le bouton \button{Effacer}.
Confirmez votre choix en cliquant sur \button{Oui} ou annuler en cliquant sur \button{Non}

\subsection{Outils d'annotation} \label{sec:annotations}

L'outil d'annotation \includegraphics[width=0.7cm, clip=true]{mActionTextAnnotation} dans la barre d'outils d'attribut fournit la possibilité de placer du texte formaté dans des phylactères sur la carte. Sélectionnez l'outil d'annotation puis cliquez sur la carte. Cette action place un marqueur à l'endroit du clic et un phylactère associé.

Un double clic dans l'emprise d'une annotation (matérialisée par quatre carrés aux angles) provoque l'ouverture d'une fenêtre de dialogue avec diverses options. Il y a un éditeur de texte avec quelques options (choix de la police de caractères, de la taille, etc.), le choix de la couleur de fond du cadre, ainsi que de la couleur et de l'épaisseur du contour. Il est également possible de choisir le marqueur. Ce dernier est affiché lorsque \checkbox{Position fixe de la carte} est activé : l'annotation est associée à un endroit de la carte et en suit les déplacements. Si l'option est désactivée, la position de l'annotation est relative à l'interface graphique et n'est pas impactée par la navigation dans la carte. La bulle peut être déplacée indépendamment du marqueur. Le déplacement du marqueur affecte l'ensemble de l'annotation.